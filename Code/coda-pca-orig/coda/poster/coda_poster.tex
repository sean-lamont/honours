\documentclass{beamer}
%% Possible paper sizes: a0, a0b, a1
%% Possible orientations: portrait, landscape
%% Font sizes can be changed using the scale option.
\usepackage[size=a0,orientation=portrait,scale=1.8]{beamerposter}
\usetheme{LLT-poster}
\usecolortheme{ComingClean}
% \usecolortheme{Entrepreneur}
% \usecolortheme{ConspiciousCreep}

\newcommand{\defeq}{\stackrel{\mathrm{.}}{=}}
%\renewcommand{\Re}{\mathbb{R}}\newcommand{\Se}{\mathbb{S}} % hollow R/S
\newcommand{\Se}{\mathfrak{S}} % for frac R/S
\newcommand{\varphidagger}{\check{\varphi}}
\newcommand{\Xdagger}{\check{\mathcal{X}}}
\newcommand{\vexdagger}{\check{\ve{x}}}
\newcommand{\veydagger}{\check{\ve{y}}}
\newcommand{\XCal}{\mathcal{X}}
\usepackage{bm}
\newcommand{\ve}[1]{\bm{#1}}

\usepackage[utf8]{inputenc}
\usepackage[T1]{fontenc}
\usepackage{libertine}
\usepackage[scaled=0.92]{inconsolata}
\usepackage[libertine]{newtxmath}

%\usepackage{../morenotations}

\title{Representation Learning of Compositional Data}
\author[]{Marta Avalos-Fernandez$^\dagger$~~Richard Nock$^\ddagger$$^\S$$^\natural$~~Cheng Soon Ong$^\ddagger$$^\S$~~Julien Rouar$^\dagger$~~Ke Sun$^\ddagger$}
% Optional foot image \footimage{\includegraphics[width=4cm]{IMG_1934.jpg}}
\institute{$^\dagger$Universit\'e de Bordeaux%
\hspace{2em}$^\ddagger$Data61%
\hspace{2em}$^\S$the Australian National University
\hspace{2em}$^\natural$the University of Sydney}

\begin{document}
\begin{frame}[fragile]\centering

\begin{columns}[T]

%%%% First Column
\begin{column}{.46\textwidth}
\begin{block}{Introduction}
\end{block}

\begin{block}{}
\end{block}

\begin{block}{}
\end{block}
\end{column}

%%%% Second Column
\begin{column}{.46\textwidth}

\begin{block}{}
\end{block}

\begin{block}{}
\end{block}
\end{column}
\end{columns}

\begin{block}{}
\begin{theorem}[Scaled Bregman Theorem with Remainder]\label{th00}
For any 
$\varphi: \XCal \rightarrow \Re$ and $g : \XCal \rightarrow \Re_*$ ($\Re_*=\Re\setminus\{0\}$)
that are both differentiable, denoting
\begin{equation*}
  \check{\ve{x}} \defeq \frac{\ve{x}}{g(\ve{x})}
  \quad\text{and}\quad
  \check{\varphi}(\ve{x}) \defeq 
  %g(\ve{x}) \cdot {\varphi}(\check{\ve{x}})  =  
  g(\ve{x}) \cdot {\varphi}\left( \frac{\ve{x}}{g(\ve{x})} \right) \:\:,\label{defCHECKPHI}
\end{equation*}
the following holds true:
  \begin{equation*}
  g(\ve{x}) \cdot D_{\varphi}\left( \vexdagger \,\Vert\, \veydagger \right)
  = D_{\varphidagger}\left( \ve{x} \,\Vert\, \ve{y}\right)
  + R_{\varphi, g}(\ve{x}\,\Vert\,\ve{y})\:,\quad\forall
  \ve{x}, \ve{y}\in \XCal\:\:,\label{defdagger}
  \end{equation*}
  where $R_{\varphi, g}(\ve{x}\,\Vert\,\ve{y}) \defeq \varphi^\star\left(\nabla\varphi(\veydagger)
  \right)\cdot D_g(\ve{x}\,\Vert\,\ve{y})$ is called the \emph{remainder}.
\end{theorem}
\end{block}

\begin{columns}
\begin{column}{.46\textwidth}
\begin{block}{Experiments}
\end{block}
\end{column}

\begin{column}{.46\textwidth}
\begin{block}{}
\end{block}
\begin{block}{}
\end{block}
\end{column}
\end{columns}

\end{frame}
\end{document}
