\chapter*{Abstract}
\addcontentsline{toc}{chapter}{Abstract}
\vspace{-1em}
  Compositional Data Analysis (CoDA) is the study of data which can be interpreted as ratios or counts. Data of this type manifests in many forms, from microbiome species counts to chemical compositions to wealth distributions. Compositional data is known to exist in a simplex, and so many standard statistical analysis methods (which assume Euclidean geometry) are unsuitable. The goal of this thesis is to expand on current dimensionality reduction research, and investigate the use of jointly optimising supervised learning  loss with the addition of a reconstruction error term. 
  
  There has been some progress in the analysis of this data, most of which centres on a log transformation which embeds the simplex in Euclidean space.  

%%% Local Variables: 
%%% mode: latex
%%% TeX-master: "paper"
%%% End: 